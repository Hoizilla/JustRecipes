%!TEX TS-program = xelatex                                              % 使用xelatex进行处理
%!TEX encoding = UTF-8 Unicode                                          % 使用UTF-8, Unicode

\documentclass[12pt, a4paper]{ctexart}

%%%%%%%%%%%%%%%%%%% 使用包 %%%%%%%%%%%%%%%%%%%

\usepackage{float}
\usepackage{titlesec,shorttoc}
\usepackage{indentfirst}                                                % 首行缩进
\usepackage[colorlinks=true]{hyperref}
\usepackage{fontspec,xunicode,xltxtra}
\usepackage[top=1in,bottom=1in,left=1.25in,right=1.25in]{geometry}
\usepackage{graphicx}
\usepackage{enumitem}
\usepackage{color}
\usepackage{xcolor}
\usepackage{multirow}
\usepackage{url}
\usepackage{bm}
\usepackage[most]{tcolorbox}
\usepackage{fancyhdr}
\usepackage{changepage}
\usepackage{natbib}
\setcounter{secnumdepth}{2}
\setcounter{tocdepth}{2}

%%%%%%%%%%%%%%%%%%% 设置字体 %%%%%%%%%%%%%%%%%%%

\defaultfontfeatures{Mapping=tex-text}
\setmainfont{STKaiti}
\setsansfont[Scale=MatchLowercase]{Gill Sans}
\setmonofont[Scale=MatchLowercase]{Menlo}
\setCJKmainfont{STKaiti}

%%%%%%%%%%%%%%%%%%% 样式设置 %%%%%%%%%%%%%%%%%%%

\hypersetup{urlcolor=Gray, citecolor=Gray}                              % 设置超链接和引用的颜色

\setlength{\parindent}{2em}                                             % 首行空两字
\addtolength{\parskip}{.4em}

%%%%%%%%%%%%%%%%%%% 开始文档 %%%%%%%%%%%%%%%%%%%

\begin{document}

\title{JustRecipes}
\author{\bfseries 笈川伊織\footnote{https://github.com/IoriOikawa/JustRecipes}}
\date{}

\maketitle
\setcounter{secnumdepth}{2}
\tableofcontents

%%%%%%%%%%%%%%%%%%% Beverage %%%%%%%%%%%%%%%%%%%
\newpage
\section{Beverage}

%%%%%%%%%%%%%%%%%%% Beverage - 懒人冷萃咖啡 %%%%%%%%%%%%%%%%%%%
\subsection{懒人冷萃咖啡}
\subsubsection{准备}
\begin{enumerate}
    \item{一个干净的容器w}
    \item{冰箱w}
    \item{咖啡粉w}
    \item{法压壶w}
\end{enumerate}

\subsubsection{步骤}
\begin{enumerate}
    \item{将3份咖啡粉倒入容器}
    \item{向容器中加7份水}
    \item{再在容器中加3份咖啡}
    \item{不许摇晃容器,放入冰箱,温度调至0~4度}
    \item{24~48小时后拿出容器,过法压壶}
    \item{0~4度且未加糖时,容器可以保存两星期}
\end{enumerate}

\subsubsection{PS}
推荐配糖浆及冰块饮用 不喜欢咖啡的酸味的话超级推荐

\subsubsection{PPS}
注意饮用量,当心手抖w

%%%%%%%%%%%%%%%%%%% Domintory Recipes %%%%%%%%%%%%%%%%%%%
\newpage
\section{Domintory Recipes}

%%%%%%%%%%%%%%%%%%% Domintory Recipes - 宿舍用火锅 %%%%%%%%%%%%%%%%%%%
\subsection{宿舍用火锅}
\subsubsection{准备}
\begin{enumerate}
    \item{一个保温杯w}
    \item{一个水壶w}
\end{enumerate}

\subsubsection{步骤}
\begin{enumerate}
    \item{底料放水壶里热一下拿出来,算是第一锅}
    \item{蓝后第二锅用水壶}
    \item{加水放次的}
    \item{涮之}
    \item{涮好的放保温杯蘸底料}
\end{enumerate}

\subsubsection{PS}
害怕辣或者想蘸底料就单独拿一个一次性杯子或者碗放底料就好

\subsubsection{PPS}
虽然大块的东西像是脑花不一定能入味儿倒是

\end{document}
